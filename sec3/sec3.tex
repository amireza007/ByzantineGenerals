\documentclass[../document.tex]{subfiles}
%\documentclass{beamer}
\usepackage{amssymb,amsmath,amsthm}
\usepackage{graphicx}
\usepackage{tikz}
%\usetheme{Singapore}
\usetheme{Boadilla}
\usecolortheme{rose}

\usetikzlibrary{tikzmark}
\usepackage{colortbl}
\usepackage{graphicx}
\usepackage{pdfpages}
\tikzstyle{every picture}+=[remember picture,baseline]
\tikzstyle{every node}+=[inner sep=0pt,anchor=base,
minimum width=1.5cm,align=center,text depth=.25ex,outer sep=1.5pt]
\tikzstyle{every path}+=[thick, rounded corners]
%\setframetemplate{frametitle}[default][center]

%%%%%%%%%%%%%%%%%%%% VERY IMPORTANT 
%very useful way to add notes to Beamer
%\setbeameroption{show notes on second screen=right}
\setbeamertemplate{note page} 
{ 
	\insertslideintonotes{0.65} 
	\rule{\textwidth}{0.1pt} 
	\color{blue} \small
	\insertnote 
}
\begin{document}
	
\begin{frame}{A review of the Byzantine problem}
	\begin{itemize}
		\item <+-> There is an \textbf{enemy city} and a group of \textbf{General $i$}, each deciding to reach \textbf{an agreed upon plan} (which is the exact definition of \textbf{consensus}) to whether \textit{Attack} or \textit{Retreat}
		\item <+-> and each general $i$ is equipped with a messaging method for sending value $v(i)(\in \{\text{Attack, Retreat}\})$ for communicating with each other
		\item <+-> AND there are a \large bunch of \textbf{traitorous generals} \normalsize sending \textbf{conflicting} messages, aim to prevent \textbf{loyal generals to reach a plan}
		\item <+-> In fact, they send \textbf{arbitrary messages} to other generals.
		\item <+-> In order for them to a reach consensus, two conditions must be satisfied:
		\begin{enumerate}
			\item <+-> Every loyal general must obtain the same information $v(1), v(2) \dots, v(n)$
			\item <.-> The value sent by a loyal general should be used by all loyal generals
		\end{enumerate}
	\end{itemize} 
\end{frame}

%%%%%%%%%%%%%%%%%%

\begin{frame}{A review of the Byzantine problem}

\end{frame}

\begin{frame}{A review of the Byzantine problem \textit{cont.}}
	\uncover <+->{How should the generals send their messages?\newline}
	\uncover<+-> {Let's examine how \textbf{a single general $i$ should send the message $v(i)$} that is formally defined by:}
	\alt <+>{\textit{why??}}{}
	\uncover<+->{\small(which is made by grouping generals into two groups, namely \textbf{commander and lieutenant generals})}
	\uncover <.->{\begin{definition} [Byzantine Generals Problem]
			A \textbf{commanding general} must send an order to his \textbf{$n-1$ lieutenant generals} s.t.:\newline
			IC1. All loyal lieutenants obey the same order.\newline
			IC2. If the commanding general is loyal, then every loyal lieutenant obeys the order he sends.
	\end{definition}}
	\uncover<+->{\small Note that IC2 $\implies$ IC1\newline}
	\uncover<+> {\textbf{*} Also note that,E.g. when \underline{General $n$} sends \underline{$v(n)$  lieutenant $n-1$} retrieves a message from \underline{$n-2$ generals} and then apply a function $Majority(v(1),v(2),\dots,v(n-2))$ and adds $v(n)$ to the list $V_1$} 
\end{frame}

%%%%%%%%%%%%

\begin{frame}{A review of the Byzantine problem \textit{cont.}}
	\uncover<+->{Lamport gave a recursive algorithm based on \textbf{majority function} for the mentioned problem in case of having \textbf{oral messages} whose content is solely managed by the sender.}
	\uncover<+->{\newline \newline Unfortunately, the algorithm there are two problems concerning this algorithm: }
		\begin{enumerate}
			\item \textbf{It is expensive $\rightarrow O(n!)$}
			\item It only works only, in case of having $m$ traitors, for $n\ge 3m+1$ generals \newline
		\end{enumerate}
	
	\uncover<+->{\begin{center}
			In order to deal with problem 2, he proposed an algorithm based on \textbf{unforgeable messages.}
	\end{center}}
\end{frame}
\end{document}