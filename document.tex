\documentclass[handout]{beamer}
\usepackage{amssymb,amsmath,amsthm}
\usepackage{graphicx}
\usepackage{tikz}
%\usetheme{Singapore}
\usetheme{Boadilla}
\usecolortheme{rose}

\usetikzlibrary{tikzmark}
\usepackage{colortbl}
\usepackage{graphicx}
\usepackage{pdfpages}
\tikzstyle{every picture}+=[remember picture,baseline]
\tikzstyle{every node}+=[inner sep=0pt,anchor=base,
minimum width=1.5cm,align=center,text depth=.25ex,outer sep=1.5pt]
\tikzstyle{every path}+=[thick, rounded corners]
%\setframetemplate{frametitle}[default][center]

%%%%%%%%%%%%%%%%%%%% VERY IMPORTANT 
%very useful way to add notes to Beamer
%\setbeameroption{show notes on second screen=right}
\setbeamertemplate{note page} 
{ 
	\insertslideintonotes{0.65} 
	\rule{\textwidth}{0.1pt} 
	\color{blue} \small
	\insertnote 
}

%%%%%%%%%%% for continuing slides enumeration
\newcounter{saveenumi}
\newcommand{\seti}{\setcounter{saveenumi}{\value{enumi}}}
\newcommand{\conti}{\setcounter{enumi}{\value{saveenumi}}}

\resetcounteronoverlays{saveenumi}
\usepackage{mathtools}

\usepackage{subfiles}

\title{Byzantine Generals problem}
\author{Amirreza Taghizadeh}

\begin{document}
	\maketitle
	\AtBeginEnvironment{frame}{\setcounter{footnote}{0}}		%for reseting the footnote counter 
\begin{frame}

 In this presentation, we aim to discuss a rproblem in distributed systems known as "\textbf{Byzantine generals problem"} proposed by \textbf{Leslie Lamport.}

\end{frame}

%%%%%%%%

\begin{frame}{Outline}
	\tableofcontents
\end{frame}
% Current section
\AtBeginSection[ ]
{
	\begin{frame}{Outline}
		\tableofcontents[currentsection]
	\end{frame}
}
\section{Brief history of Lamport's works}
\subfile{./sec1/sec1.tex}
%%%%%%%%%

\section{Origination of Bezyntine Generals problem}
\subfile{./sec2/sec2.tex}
%%%%%%%%%%%%%%%


\section{A review of the Byzantine problem}
\subfile{./sec3/sec3.tex}
%%%%%%%%%%%%

\section{Signed messages}
\subfile{./sec4/sec4.tex}

%%%%%%%%%%%%%%

\section{Practical Byzantine-fault tolerance}
\subfile{./sec5/sec5.tex}

%%%%%%%%%%%%%%%

\section{Topics Discussed}
\begin{frame}{\insertsection}
	\begin{itemize}
		\item <+-> Appreciation of Lamport's Other Works: \LaTeX, Paxos, Bakery algorithm, One-way authentication,\textbf{ Time clocks in Asynchronous Distributed systems} etc.
		\item<+-> What is the Byzantine generals problem and how does it relate to the problem of solving \textbf{consensus}? 
		\item<+-> Using Oral messaging to solve Byzantine-fault problem + \textbf{its weaknesses}$\rightarrow n \ge 3m+1$ 
		\item<+-> What is the structure of \textbf{signed and unforgeable messages} and How does it solve Oral Messaging?
		\item<+-> Defining time with \textbf{logical clocks} in a system \textbf{without a global physical clock} (Asynchronous Systems)
		\item<+-> Is there a more efficient algorithm for Solving Byzantine failure in asynchronous System? $\rightarrow$ Practical Byzantine-fault Tolerant Algorithm
	\end{itemize}
\end{frame}

%%%%%%%%%%%%%%

\begin{frame}
	\frametitle{References:}
\begin{thebibliography}{10}
	\footnotesize
	\bibitem{lamport82}
	\alert{L. Lamport, R. Shostak, and M. Pease}
	\newblock  {The Byzantine Generals Problem}
	\newblock {\em ACM Transactions on Programming Languages and Systems (1982)}.
	
	\bibitem{FLP}
	\alert{M. Fishcher, N. Lynch, and M. Paterson}
	\newblock {Impossibility of Distributed Consensus with One Faulty Process}
	\newblock {\em Journal of the Association for Computing Machinery, Vol. 32, No. 2, (1985)}.
	
	\setbeamertemplate{bibliography item}[book]
	\bibitem{B} 
	\alert{C. Cachin, R. Guerraoui and L. Rodrigues}
	\newblock {Introduction to Reliable and Secure Distributed Programming, 2nd Edition}
	\newblock {\em Springer (2011), p. 44-48}.
	
	\bibitem{C}
	\alert{A. Silberschatz, H. Korth, and S. Sudarshan}
	\newblock{Database System Concepts, 7th Edition}
	\newblock{\em McGraw-Hill Education (2020), p. 965}
	
	\setbeamertemplate{bibliography item}[article]
	\bibitem{D}
	\alert{L. Lamport}
	\newblock{Time, Clocks, and the Ordering of Events in a Distributed System}
	\newblock{\em Journal of the Association for Computing Machinery, Vol. 21, No. 7, (1978)}
\end{thebibliography}
\end{frame}

\begin{frame}
	\frametitle{References:}
	\begin{thebibliography}{10}
	\footnotesize
	\bibitem{E}
	\alert{M. Castro, B. Liskov}
	\newblock{Practical Byzantine Fault Tolerance}
	\newblock{\em Third Symposium on Operating Systems Design and Implementation, New Orleans, (1999)}
	
	\bibitem{F}
	\alert{W. Diffie and M. Hellman}
	\newblock{New Directions in Cryptography}
	\newblock{\em IEEE Transactions on Information Theory, Vol. IT-22, No. 6 (1976)}
	
	\bibitem{G}
	\alert{L. Lamport}
cd	\newblock{Constructing Digital Signatures from a One Way Function}
	\newblock{\em SRI International (1979)}
	
	\bibitem{H}
	\alert{F. A. Parand}
	\newblock{Failure and Consensus}
	\newblock{\em (.ppt) file used as teaching material in Distributed Systems class by Prof. Parand, Allameh Tabatabai University}
\end{thebibliography}
\end{frame}

%%%%%%%%%%%%%%%

\begin{frame}
	\Large \centering Any Questions?
\end{frame}
\end{document}